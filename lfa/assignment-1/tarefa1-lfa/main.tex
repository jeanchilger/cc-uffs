\documentclass[a4paper, 12pt]{article}

\usepackage[portuges]{babel}
\usepackage[utf8]{inputenc}
\usepackage{amsmath}
\usepackage{indentfirst}
\usepackage{graphicx}
\usepackage{multicol,lipsum}

\usepackage{hyperref}   % adiciona links na TOC e nas citações

% Renomeia o título da TOC
\addto\captionsportuges{
  \renewcommand{\contentsname}
    {Sumário}
}

%%%%%%%%%%%%%%%%%%%%%%%%%%%%%%%%%%%%%%%%%%%%%%%%%%%%%%%%%%
% INÍCIO DO DOCUMENTO
%%%%%%%%%%%%%%%%%%%%%%%%%%%%%%%%%%%%%%%%%%%%%%%%%%%%%%%%%%

\begin{document}
%\maketitle

%%%%%%%%%%%%%%%%%%%%%%%%%%%%%%%%%%%%%%%%%%%%%%%%%%%%%%%%%%%
% CAPA
%%%%%%%%%%%%%%%%%%%%%%%%%%%%%%%%%%%%%%%%%%%%%%%%%%%%%%%%%%%
\begin{titlepage}
	\begin{center}
    	\begin{figure}[!ht]
        	\centering
        	\includegraphics[width=3cm]{./imgs/uffs.png}
    	\end{figure}
    
    	\Huge{Universidade Federal da Fronteira Sul}\\
    	\large{Campus Chapecó}\\ 
    	\large{Bacharelado em Ciência da Computação}\\ 
    	
    	\vspace{15pt}
        \vspace{95pt}
        
        \textbf{\LARGE{Exercícios indicados nas aulas 03 e 04}}\\
    	
        \vspace{3,5cm}
	\end{center}
	
	\begin{flushleft}
	    \begin{tabbing}
			\textbf{Aluno:} Jean Carlo Hilger \\
			\textbf{Professor:} Andrei Braga \\
        \end{tabbing}
    \end{flushleft}
	
	\vspace{1cm}
	
	\begin{center}
		\vspace{\fill}
		Chapecó, março\\
		2021
	\end{center}
\end{titlepage}


%%%%%%%%%%%%%%%%%%%%%%%%%%%%%%%%%%%%%%%%%%%%%%%%%%%%%%%%%%%
% SUMÁRIO
%%%%%%%%%%%%%%%%%%%%%%%%%%%%%%%%%%%%%%%%%%%%%%%%%%%%%%%%%%%
\newpage
\tableofcontents
\thispagestyle{empty}

\newpage
\pagenumbering{arabic}

%%%%%%%%%%%%%%%%%%%%%%%%%%%%%%%%%%%%%%%%%%%%%%%%%%%%%%%%%%%
% Exercício 1
%%%%%%%%%%%%%%%%%%%%%%%%%%%%%%%%%%%%%%%%%%%%%%%%%%%%%%%%%%%
\newpage
\section{Exercício 1}

A Figura \ref{fig:m1_m2} representa os diagramas de estado de dois autômatos finitos $M_1$ e $M_2$. Responda às questões sobre cada um desses autômatos.

\begin{figure}[!ht]
    \centering
    \includegraphics[width=10cm]{./imgs/img1.png}
    \caption{Diagramas dos autômatos finitos $M_1$ e $M_2$.}
    \label{fig:m1_m2}
\end{figure}

\subsection{Autômato M1}

% Pergunstas e Respostas
\begin{itemize}
    \item \textbf{Qual é o estado inicial?}
    \begin{itemize}
        \item Resposta
    \end{itemize}
    
    
    \item \textbf{Qual é o conjunto de estados de aceitação?}
    \begin{itemize}
        \item Resposta
    \end{itemize}
    
    
    \item \textbf{Por qual sequência de estados o autômato passa na entrada aabb?}
    \begin{itemize}
        \item Resposta
    \end{itemize}
    
    
    \item \textbf{O autômato aceita a string aabb?}
    \begin{itemize}
        \item Resposta
    \end{itemize}
    
    
    \item \textbf{O autômato aceita a string $\epsilon$?}
    \begin{itemize}
        \item Resposta
    \end{itemize}
\end{itemize}

\subsection{Autômato M2}
% Pergunstas e Respostas
\begin{itemize}
    \item \textbf{Qual é o estado inicial?}
    \begin{itemize}
        \item Resposta
    \end{itemize}
    
    
    \item \textbf{Qual é o conjunto de estados de aceitação?}
    \begin{itemize}
        \item Resposta
    \end{itemize}
    
    
    \item \textbf{Por qual sequência de estados o autômato passa na entrada aabb?}
    \begin{itemize}
        \item Resposta
    \end{itemize}
    
    
    \item \textbf{O autômato aceita a string aabb?}
    \begin{itemize}
        \item Resposta
    \end{itemize}
    
    
    \item \textbf{O autômato aceita a string $\epsilon$?}
    \begin{itemize}
        \item Resposta
    \end{itemize}
\end{itemize}

%%%%%%%%%%%%%%%%%%%%%%%%%%%%%%%%%%%%%%%%%%%%%%%%%%%%%%%%%%%
% Exercício 2
%%%%%%%%%%%%%%%%%%%%%%%%%%%%%%%%%%%%%%%%%%%%%%%%%%%%%%%%%%%
\newpage
\section{Exercício 2}

Dê a definição formal (através de uma 5-upla) dos autômatos considerados no exercício anterior.

\subsection{Autômato M1}

\subsection{Autômato M2}

%%%%%%%%%%%%%%%%%%%%%%%%%%%%%%%%%%%%%%%%%%%%%%%%%%%%%%%%%%%
% Exercício 3
%%%%%%%%%%%%%%%%%%%%%%%%%%%%%%%%%%%%%%%%%%%%%%%%%%%%%%%%%%%
\newpage
\section{Exercício 3}

A definição formal de um autômato finito $M$ é 
$$(\{q_1, q_2, q_3, q_4, q_5\}, \{u, d\}, \delta, q_3, \{q_3\})$$

Onde $\alpha$ é dado pela Tabela \ref{fig:table_delta} Dê o diagrama de estados desse autômato.

\begin{figure}[!ht]
    \centering
    \includegraphics[width=3cm]{./imgs/table_delta.png}
    \caption{Tabela da função de transição $\delta$.}
    \label{fig:table_delta}
\end{figure}


% \begin{table}[]
%     \centering
%     \begin{tabular}{c|c|c}
%         u & d \\
%          & 
%     \end{tabular}
%     \caption{Caption}
%     \label{tab:my_label}
% \end{table}


\end{document}



