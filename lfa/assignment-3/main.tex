\documentclass[a4paper, 12pt]{article}

\usepackage[portuges]{babel}
\usepackage[utf8]{inputenc}
\usepackage{amsmath}
\usepackage{indentfirst}
\usepackage{graphicx}
\usepackage{multicol,lipsum}

\usepackage{hyperref}   % adiciona links na TOC e nas citações

% Renomeia o título da TOC
\addto\captionsportuges{
  \renewcommand{\contentsname}
    {Sumário}
}

%%%%%%%%%%%%%%%%%%%%%%%%%%%%%%%%%%%%%%%%%%%%%%%%%%%%%%%%%%
% INÍCIO DO DOCUMENTO
%%%%%%%%%%%%%%%%%%%%%%%%%%%%%%%%%%%%%%%%%%%%%%%%%%%%%%%%%%

\begin{document}
%\maketitle

%%%%%%%%%%%%%%%%%%%%%%%%%%%%%%%%%%%%%%%%%%%%%%%%%%%%%%%%%%%
% CAPA
%%%%%%%%%%%%%%%%%%%%%%%%%%%%%%%%%%%%%%%%%%%%%%%%%%%%%%%%%%%
\begin{titlepage}
	\begin{center}
    	\begin{figure}[!ht]
        	\centering
        	\includegraphics[width=3cm]{./imgs/uffs.png}
    	\end{figure}
    
    	\Huge{Universidade Federal da Fronteira Sul}\\
    	\large{Campus Chapecó}\\ 
    	\large{Bacharelado em Ciência da Computação}\\ 
    	
    	\vspace{15pt}
        \vspace{95pt}
        
        \textbf{\LARGE{Exercícios indicados nas aulas 07 e 08}}\\
    	
        \vspace{3,5cm}
	\end{center}
	
	\begin{flushleft}
	    \begin{tabbing}
			\textbf{Aluno:} Jean Carlo Hilger \\
			\textbf{Professor:} Andrei de Almeida Sampaio Braga \\
        \end{tabbing}
    \end{flushleft}
	
	\vspace{1cm}
	
	\begin{center}
		\vspace{\fill}
		Chapecó, março\\
		2021
	\end{center}
\end{titlepage}


%%%%%%%%%%%%%%%%%%%%%%%%%%%%%%%%%%%%%%%%%%%%%%%%%%%%%%%%%%%
% SUMÁRIO
%%%%%%%%%%%%%%%%%%%%%%%%%%%%%%%%%%%%%%%%%%%%%%%%%%%%%%%%%%%
\newpage
\tableofcontents
\thispagestyle{empty}

\newpage
\pagenumbering{arabic}

%%%%%%%%%%%%%%%%%%%%%%%%%%%%%%%%%%%%%%%%%%%%%%%%%%%%%%%%%%%
% Exercício 1
%%%%%%%%%%%%%%%%%%%%%%%%%%%%%%%%%%%%%%%%%%%%%%%%%%%%%%%%%%%
\newpage
\section{Exercício 1}

Para cada autômato finito não-determinístico abaixo, construa um autômato finito determinístico equivalente.

\subsection{AFN M1}

\begin{figure}[!ht]
    \centering
    \includegraphics[width=4cm]{./imgs/task-1.png}
    \caption{Autômato não determinístico M1.}
    \label{fig:m1_m2}
\end{figure}

\subsubsection{AFD Equivalente}

\subsection{AFN M2}

\begin{figure}[!ht]
    \centering
    \includegraphics[width=6cm]{./imgs/task-2.png}
    \caption{Autômato não determinístico M2.}
    \label{fig:m1_m2}
\end{figure}

\subsubsection{AFD Equivalente}

%%%%%%%%%%%%%%%%%%%%%%%%%%%%%%%%%%%%%%%%%%%%%%%%%%%%%%%%%%%
% Exercício 2
%%%%%%%%%%%%%%%%%%%%%%%%%%%%%%%%%%%%%%%%%%%%%%%%%%%%%%%%%%%
\newpage
\section{Exercício 2}

Para cada uma das linguagens abaixo, escreva uma expressão regular que represente a linguagem. Todas as linguagens são linguagens sobre o alfabeto $\{0, 1\}$.

\begin{enumerate}
    \item $L = \{$ $w$ $|$ $w$ começa com um 1 e termina com um 0 $\}$
    \begin{itemize}
        \item $1(1+0)^*0$.
    \end{itemize}
    
    \item $L = \{$ $w$ $|$ $w$ contém pelo menos três 1's $\}$
    \begin{itemize}
        \item $(1+0)^*1(1+0)^*1(1+0)^*1(1+0)$.
    \end{itemize}
    
    \item $L = \{$ $w$ $|$ $w$ contém a substring $0101$ $\}$
    \begin{itemize}
        \item $(1+0)^*0101(1+0)^*$.
    \end{itemize}
    
    \item $L = \{$ $w$ $|$ $w$ tem comprimento pelo menos 3 e o terceiro símbolo de $w$ é um 0 $\}$
    \begin{itemize}
        \item $(1+0)(1+0)0(1+0)^*$.
    \end{itemize}
    
    \item $L = \{$ $w$ $|$ $w$ começa com 0 e tem comprimento ímpar ou w começa com 1 e tem comprimento par $\}$
    \begin{itemize}
        \item $0((1+0)(1+0))^* + 1(1+0)((1+0)(1+0))^*$.
    \end{itemize}
    
\end{enumerate}

\end{document}
