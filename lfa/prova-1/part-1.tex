\documentclass[a4paper, 12pt]{article}

\usepackage[portuges]{babel}
\usepackage[utf8]{inputenc}
\usepackage{amsmath}
\usepackage{indentfirst}
\usepackage{graphicx}
\usepackage{multicol,lipsum}

\usepackage{hyperref}   % adiciona links na TOC e nas citações

% Renomeia o título da TOC
\addto\captionsportuges{
  \renewcommand{\contentsname}
    {Sumário}
}

%%%%%%%%%%%%%%%%%%%%%%%%%%%%%%%%%%%%%%%%%%%%%%%%%%%%%%%%%%
% INÍCIO DO DOCUMENTO
%%%%%%%%%%%%%%%%%%%%%%%%%%%%%%%%%%%%%%%%%%%%%%%%%%%%%%%%%%

\begin{document}
%\maketitle

%%%%%%%%%%%%%%%%%%%%%%%%%%%%%%%%%%%%%%%%%%%%%%%%%%%%%%%%%%%
% CAPA
%%%%%%%%%%%%%%%%%%%%%%%%%%%%%%%%%%%%%%%%%%%%%%%%%%%%%%%%%%%
\begin{titlepage}
	\begin{center}
    	\begin{figure}[!ht]
        	\centering
        	\includegraphics[width=3cm]{./imgs/uffs.png}
    	\end{figure}
    
    	\Huge{Universidade Federal da Fronteira Sul}\\
    	\large{Campus Chapecó}\\ 
    	\large{Bacharelado em Ciência da Computação}\\ 
    	
    	\vspace{15pt}
        \vspace{95pt}
        
        \textbf{\LARGE{Parte 1 da Prova de LFA}}\\
    	
        \vspace{3,5cm}
	\end{center}
	
	\begin{flushleft}
	    \begin{tabbing}
			\textbf{Aluno:} Jean Carlo Hilger \\
			\textbf{Matrícula:} 1811100018 \\
			\textbf{Professor:} Andrei de Almeida Sampaio Braga \\
        \end{tabbing}
    \end{flushleft}
	
	\vspace{1cm}
	
	\begin{center}
		\vspace{\fill}
		Chapecó, abril\\
		2021
	\end{center}
\end{titlepage}


%%%%%%%%%%%%%%%%%%%%%%%%%%%%%%%%%%%%%%%%%%%%%%%%%%%%%%%%%%%
% SUMÁRIO
%%%%%%%%%%%%%%%%%%%%%%%%%%%%%%%%%%%%%%%%%%%%%%%%%%%%%%%%%%%
\newpage
\tableofcontents
\thispagestyle{empty}

\newpage
\pagenumbering{arabic}

\section{Questão 1}
\textbf{Resposta:} D - aceita / 101010 / q0 

\section{Questão 2}
\textbf{Resposta:} C - consista em um ou mais a’s seguidos de um ou mais b’s

\section{Questão 3}
\textbf{Resposta:} A - I é a única afirmacão verdadeira

\section{Questão 4}
\textbf{Resposta:} B - II e IV são as únicas afirmações verdadeiras


\end{document}
